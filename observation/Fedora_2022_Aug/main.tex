%!TEX program = xelatex
%!BIB program = bibtex

\documentclass[cn,black,12pt,normal]{elegantnote}
\usepackage{float}
\usepackage{hyperref}
\usepackage{amsmath}
\usepackage{amssymb}
\usepackage{pdfpages}

\hypersetup{hidelinks,
backref=true,
pagebackref=true,
hyperindex=true,
breaklinks=true,
colorlinks=true,%linkcolor=black,
urlcolor=blue,
bookmarks=true,
bookmarksopen=false,
pdftitle={Title},
pdfauthor={Author}}


\lstset{%
basicstyle=\linespread{0.8}\tt,
frame=single, %把代码用带有阴影的框圈起来
breaklines=true, %对过长的代码自动换行
}

\newcommand{\setParDef}{\setlength {\parskip} {0pt} }
\newcommand{\upcite}[1]{\textsuperscript{\textsuperscript{\cite{#1}}}}

\title{Fedora $\times$ RISC-V \\Aug 2022 一页简报}
\author{『实习生』姜文渊}
\institute{BJ-67 Tarsier-PLCT}
%\version{0.01}
\date{2022 年 8 月 30 日}
% \date{2022 年 7 月 13 日}
\begin{document}
% \setParDef
\maketitle

下面的内容来自 \url{scm-commits@lists.fedoraproject.org},未进行完整的实机验证。

\section*{OpenJDK}

\paragraph*{java-latest-openjdk (rawhide) "Update to RC version of OpenJDK 19 (..more)"} 目前的 Fedora rawhide 中已经加入了 \lstinline{openjdk-jdk19u-jdk-19+36.tar.xz},官方文档中宣称添加了 \lstinline{Linux/RISC-V Port}。感谢开发者 pagure 的 push 。\href{https://lists.fedoraproject.org/archives/list/scm-commits@lists.fedoraproject.org/message/UJ2DXOCMFG6H6WKK3MCKQYH66PCLMOKJ/}{SEE HERE}

\section*{LuaJIT}

\paragraph*{luajit (rawhide)(f37)(f36) "Update to latest luajit v2.1 git version"} 目前的 Fedora 的 rawhide 、 f37 和 f36 中已经更新到了 \lstinline{luajit v2.1 git version},官方文档中宣称添加了 \lstinline{RISC-V 64 RVA22+ TBA} 的支持。感谢开发者 asn 的 push 。\href{https://lists.fedoraproject.org/archives/list/scm-commits@lists.fedoraproject.org/message/T4ABSC6XFUC2CXRGISJPM7GI7NQIMSVK/}{SEE HERE}

\section*{mold}

\paragraph*{mold (rawhide) "Bump version to 1.4.1"} 目前 mold 在 Fedora 的 rawhide 中已经更新到了 \lstinline{1.4.1} 版本,官方文档中宣称其支持链接 \lstinline{riscv64} 的原生 binary。
\begin{lstlisting}[commentstyle=\color{black}]
    # mold can currently produce native binaries for these architectures only
    ExclusiveArch:	%{ix86} x86_64 %{arm} aarch64 riscv64
\end{lstlisting}
感谢开发者 sicherha 的 push 。\href{https://lists.fedoraproject.org/archives/list/scm-commits@lists.fedoraproject.org/message/WDUJUOX3XOMIVAZRMYVEGTJQADEYL6RT/}{SEE HERE}

\end{document}